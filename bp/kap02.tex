\chapter{Architektura hry}


\section{Vesmírné objekty}
Všechny vesmírné objekty mají některá data společná. Každý vesmírný objekt má souřadnice své současné polohy a také vektor rychlosti.

\subsection{Asteroidy}
Asteroidy nesou navíc informace o tom, jaké jsou velikosti a zda-li byly vytvořené nějakým z hráčů. 
Na základě těchto dvou informací je asteroidu při vytvoření přiřazen obrázek, pomocí kterého je po dobu své existence vykreslován.

\subsection{Střely}
Vystřelené střely neletí věčně, ale mají omezenou životnost kolik kroků hry budou existovat.
Tato hodnota se nastavuje z konfiguračního souboru z položky \emph{\uppercase{bullet\_life\_count}}.
V každém kroku hry se střele její živostnost sníží o jedna a pokud se dostane na nulu, tak střela bude zničena.
Střele se při vytvoření nastaví úhel, pod kterým poletí. Tento úhel je roven úhlu natočení vesmírné lodi, který měla při vystřelení.
Samozřejmě také u střely musíme evidovat, kterému z hráčů patří, toto je řešeno odkazem na objekt vesmírné lodi, která střelu vystřelila.
Jak již bylo zmíněno v předchozí kapitole, střely jsou dvojího druhu. Příznakem \emph{split} se určuje zda se jedná o střelu obyčenou nebo rozdvojovací


\subsection{Vesmírná loď}
Vesmírná loď má základní polohové informace rozšířené o úhel. Ten se s každou rotací lodě zvětší nebo zmenší o 12\textdegree.
Akcelerace funguje vektorovým sčítáním. K současnému vektoru rychlosti se přičte vektor odpovídající současnému úholu lodi.
Maximální rychlost vesmírné lodi je omezená, v případě že akcelerací vznikne vektor rychlosti, jehož délka je větší než hodnota maximální rychlosti, se směr vektoru zachová, ale požadovaně se zkrátí.

\newpage



\section{Prostředí}

Hra běží v cyklu diskrétních kroků, které dohromady simulují plynulý pohyb hry.
Herní prostředí je insipirováno projektem open ai gym od google
(viz \url{https://gym.openai.com/}). Jedním rozdílem je však přístup k vykreslování hry. V případě \emph{gym.openai} se prostředí vykresluje zavoláním metody \emph{render()} na instanci prostředí zvenku.
Já jsem zvolil přístup jiný. V případě, že chceme hru graficky zobrazovat, předáváme v kontruktoru prostředí grafický modul, který implementuje vykreslování jednotlivých typů vesmírných objektů.
A prostředí už poté objekty graficky vyresluje interně samo. Rozhodnutí, že se má grafický modul volitelně injektovat v konstruktoru a nemá být natvrdo svázen s prostředím, jsem učinil pro větší nezávislost modulů. 
Později se při práci s různými knihovnami pro evoluční algoritmy ukázalo být problematické, že bylo herní prostředí svázáno s grafickým modulem.
\par

Herní prostředí se stará o manipulací všech vesmírných objektů a akcí s nimi spojenými. 
V každém kroku dostává od hráčů akce, které chtějí provést, a prostředí na to odpovídajícím způsobem reaguje. 
Akce každého hráče z hráčů jsou pole, které obsahuje elementární možné akce:
\begin{itemize}
    \item Rotace vlevo
    \item Rotave vpravo
    \item Akcelerace
    \item Obyčejná střela
    \item Rozdvojovací střela
\end{itemize} 
Hráč může provádět více akcí najednou. Na základě přítomných elementárních akcí se provadí dané reakce.
Prostředí se stará o vesmírné objekty přímo. V případě elementárních akcí, které mění rychlost nebo orientaci vesmírné lodi, prostědí zavolá funkce, které požadované změny na vesmírné lodi provede.
A v případě elemntárních akcí střel se na základě polohy a orientace dané vesmírné lodi vytvoří nová střela, kterou opět bude spravovat právě prostředí.


\par


Jedna instance prostředí odpovídá jedné hře. Herní prostředí má dvě základní metody pro řízení hry. 
\newline 
Metoda \emph{reset()} inicializuje hru do počátečního stavu a tento stav vrátí. Tato metoda se musí zavolat před začátkem hry.
\newline 
A druhá metoda \emph{next\_step(actions\_one, actions\_two)}, která na na základě akcí hráčů, převede hru do následného stavu.
V právě této metodě je schovaná celá logika manipulace s vesmírnými objekty.

\newpage
\begin{lstlisting}[language=Python]
def next_step(self, actions_one, actions_two):
    self.step_count = self.step_count + 1
    self.reward_one = 0
    self.reward_two = 0

    self.handle_actions(actions_one, actions_two)
    self.generate_asteroid()
    self.check_collisions()
    self.move_objects()
    if self.draw_modul is not None:
        self.render()

    (game_over, player_one_won) = self.check_end()
    if not game_over:
        self.reward_one += 1
        self.reward_two += 1

    current_state = State(self.asteroids_neutral, 
                          self.rocket_one, 
                          self.asteroids_one, 
                          self.bullets_one,
                          self.rocket_two, 
                          self.asteroids_two, 
                          self.bullets_two)

    return self.step_count, \
           (game_over, player_one_won), \
           current_state, \
           (self.reward_one, self.reward_two)
\end{lstlisting}
\newpage



\subsection{Stav prostředí}
Úplná informace o všech objektech v prostředí.

\section{Agent}
Agent se na základě informace o současném stavu prostředí rozhodne o své následné akci. 


\section{Grafické prostředí}

Mezi nejvíce citované statistické články patří práce Kaplana a~Meiera a~Coxe
\citep{KaplanMeier58, Cox72}. \citet{Student08} napsal článek o~t-testu.

projektu ACCEPT jsou uvedeny v~práci \citet*{Genberget08}.
