\chapter{Architektura hry}
V úvodní kapitole jsme se seznámili s fungováním hry z uživatelského pohledu. 
Zde se pro změnu podíváme jak je hra navržena interně, jaké stavební kameny obsahuje, jak jsou reprezentovány a jaký je jejich význam.

\section{Vesmírné objekty}
Všechny vesmírné objekty mají některá data společná. Každý vesmírný objekt má souřadnice své současné polohy a vektor rychlosti.

\subsection{Asteroidy}
Asteroidy mají navíc informace o tom, jaké jsou velikosti a zdali byly vytvořené nějakým z hráčů, tedy jsou projektily, anebo byly vytvořeny jako asteroidy neutrální.
Na základě těchto dvou informací je asteroidu při vytvoření přiřazen obrázek, pomocí kterého je po dobu své existence vykreslován.

\subsection{Střely}
Vystřelené střely neletí věčně, ale mají omezenou životnost kolik kroků hry budou existovat.
Tato hodnota se nastavuje z konfiguračního souboru z položky \emph{\uppercase{bullet\_life\_count}}.
V každém kroku hry se střele její živostnost sníží o jedna a pokud se dostane na nulu, tak střela bude zničena.
Střele se při vytvoření nastaví úhel, pod kterým poletí. Tento úhel je roven úhlu natočení vesmírné lodi, který měla při vystřelení.
Přirozeně, stejně jako u asteroidů, i u střely musíme evidovat, kterému z hráčů patří, toto je řešeno odkazem na objekt vesmírné lodi, která střelu vystřelila.
Jak již bylo zmíněno v předchozí kapitole, střely jsou dvojího druhu. Příznakem \emph{split} se určuje zda se jedná o střelu obyčejnou nebo rozdvojovací


\subsection{Vesmírná loď}
Vesmírná loď má základní polohové informace rozšířené o úhel. Ten se s každou rotací lodě zvětší nebo zmenší o 12\textdegree.
Akcelerace je realizována pomocí vektorového sčítání. K současnému vektoru rychlosti se přičte vektor odpovídající současnému natočení lodi.
Maximální rychlost vesmírné lodi je omezená.
V případě že akcelerací vznikne vektor rychlosti, jehož délka je větší než hodnota maximální rychlosti, tak dojde k jeho zkrácení.
Směr vektoru se zachová, ale je délka bude zkrácena na maximální možnou délku.

\newpage



\section{Prostředí}

Hra běží v cyklu diskrétních kroků, které dohromady simulují plynulý pohyb hry.
Herní prostředí je inspirováno projektem \emph{open ai gym} od Google
(\cite{openAiGym}). Jedním rozdílem je však přístup k vykreslování hry. V případě \emph{open ai gym} se prostředí vykresluje zavoláním metody \emph{render()} na instanci prostředí zvenku.
Já jsem zvolil přístup jiný. V případě, že chceme hru graficky zobrazovat, předáme v konstruktoru prostředí grafický modul, který vykreslování vesmírných objektů implementuje.
A prostředí už poté objekty graficky vykresluje interně samo. 
Rozhodnutí, že se má grafický modul volitelně injektovat v konstruktoru a nemá být natvrdo svázán s prostředím, jsem učinil z důvodu větší nezávislosti modulů. 
Při další práci s knihovnami pro evoluční algoritmy se ukázalo být pevné svázání herního prostředí s grafickým modulem problematické.
\par

Herní prostředí se stará o manipulaci všech vesmírných objektů a akcí s nimi spojenými. 
V každém kroku dostává od hráčů akce, které chtějí provést, a prostředí na to odpovídajícím způsobem reaguje. 
Akce každého hráče jsou reprezentovány polem, které obsahuje elementární možné akce:
\begin{itemize}
    \item Rotace vlevo
    \item Rotace vpravo
    \item Akcelerace
    \item Obyčejná střela
    \item Rozdvojovací střela
    \item Prázdná akce
\end{itemize} 
Hráč může provádět více akcí najednou. Na základě přítomných elementárních akcí se provádí dané reakce.
Prostředí se stará o vesmírné objekty přímo. V případě elementárních akcí, které mění rychlost nebo orientaci vesmírné lodi, prostředí zavolá funkce, které požadované změny na vesmírné lodi provede.
A v případě elementárních akcí střel se na základě polohy a orientace dané vesmírné lodi vytvoří nová střela, kterou opět bude mít ve správě právě prostředí.

\par
Hra, jak již bylo řečeno, má být konečná, toho je docíleno narůstajícím počtem asteroidů. Toto inkrementální generování asteroidů je také zodpovědností herního prostředí.
Herní prostředí si pamatuje počet kroků, který uběhl od posledního vytvoření asteroidu. Pokud tento počet překročí danou mez, tak prostředí vytvoří nový asteroid.
Postupného nárůstu nových asteroidů je docíleno inkrementálním snižováním této meze.

\par
Další důležitou funkcí herního prostředí je kontrola srážek. Všechny objekty jsou prostorově reprezentovány jako kruhy s danými poloměry.
Postupně se prochází všechny objekty, u kterých nás zajímají srážky a Euklidovskou metrikou se kontroluje, zda od sebe nejsou vzdáleny méně, než je součet jejich poloměrů.
V případě srážky se prostředí postará o správnou reakci - zničení nebo změnu sražených objektů a případně vytvoření nových objektů vzniklých srážkou.

\par
V rámci srážek se upravují také odměny jednotlivých hráčů. Odměna je hodnota, která vyjadřuje, jak úspěšný byl tento krok pro každého z hráčů.
V každém kroku, který hráč přežil, obdrží odměnu hodnoty 1. Existují ale konkrétní srážky objektů, které hodnotu odměny mohou změnit.
V případě, že hráč sestřelil nepřátelský asteroid, nebo svým asteroidem srazil nepřátelskou loď, se výše odměny zvýší. 
Naopak, pokud byla jeho vesmírná loď zasažena nepřátelským asteroidem, je hodnota odměny snížena. 
Koncept odměn nijak neovlivňuje samotný běh hry, ale bude se nám hodit v dalších kapitolách v umělé inteligenci.

\par
Nezmínili jsme zatím pohyb objektů. I ten přirozeně spadá do logiky herního prostředí. 
Zde se prochází seznamy všech vesmírných objektů a pohyb se provede přičtením jejich vektoru rychlosti k současné poloze. Po provedení pohybu se u všech objektů provede kontrola, zda se náhodou nevyskytují mimo herní prostor. 
Pokud toto nastane, tak jsou příslušné objekty vráceny zpět do prostoru na své odpovídající místo.


\par
Pokud byl při vytvoření prostředí předán grafický modul, tak se prostředí postará i o vykreslení vesmírných objektů.
Pro všechny vesmírné objekty se na grafickém modulu zavolá příslušná metoda pro jejich vykreslení. 
Způsob implementace vykreslení jednotlivých objektů již není odpovědností herního prostředí, ale o to se stará grafický modul.

\par
Poslední zatím nezmíněnou funkcí herního prostředí je kontrola, zda hra neskončila. Na konci kroku herní prostředí kontroluje, zda mají oba hráči kladný počet životů a případě hru ukončí a informuje agenty o konečném stavu.

\par




Jedna instance prostředí odpovídá jedné hře. Herní prostředí má dvě základní metody pro řízení hry. 
\newline 
Metoda \emph{reset()} inicializuje hru do počátečního stavu a tento stav vrátí. Tato metoda se musí zavolat před začátkem hry.
\newline 
A druhá metoda \emph{next\_step(actions\_one, actions\_two)}, která na na základě akcí hráčů, převede hru do následného stavu.
Právě v této metodě je schovaná celá logika manipulace s vesmírnými objekty popsána výše.
\newline
\begin{lstlisting}[language=Python]
def next_step(actions_one, actions_two):
    handle_actions(actions_one, actions_two)
    generate_asteroid()
    check_collisions()
    move_objects()
    draw_objects()
    check_end()
    return step_count, game_over, state, reward    
\end{lstlisting}
\newpage



\subsection{Stav prostředí}
Herní prostředí vrací po každém kroku současný stav hry. Stav hry se skládá ze seznamů všech vesmírných objektů včetně kompletních informací o nich.
Pravděpodobně by bylo možné vracet i méně obsáhlou informaci o současném stavu hry. 
Avšak motivací pro mě bylo předávat kompletní informace o všech objektech a následně až v jednotlivých experimentech volit pro rozhodování jen omezené, či z těchto kompletních informací vyextrahované, informace o stavu.
Mou snahou bylo, aby herní prostředí neomezovalo agenty v jejich rozhodování a poskytovalo jim kompletní informace.


\section{Agent}
Agent je ústřední postavou celé hry a obzvláště v dalších kapitolách pro nás bude nejzajímavějším předmětem zájmu.
Je to právě toto místo, kde budeme později mluvit o umělé inteligenci. 
V případě agenta, který je ovládán lidským hráčem, se agent, respektive člověk, který jej ovládá, neřídí datovou reprezentací stavu, tak jak jej obdržel od herního prostředí.
Ale rozhoduje se na základě toho, jak hráč vizuálně vnímá herní průběh a příkazy k provedení jednotlivých akcí udává ovládáním kláves na klávesnici.
Lidský hráč pro nás ale nebude primárním předmětem zájmu, my se budeme spíše soustředit na agenty strojové.
\par
Každý agent musí implementovat jedinou metodu \emph{choose\_actions(state)}, která musí vracet akce, které chce agent v současném stavu hry vykonat. Úkolem agenta je, na základě obdrženého stavu, zvolit akci, kterou chce provést.
A právě tento rozhodovací problém pro nás bude v dalších kapitolách předmětem experimentování s různými abstrakcemi a přístupy umělé inteligence. 


\section{Grafické prostředí}
Pro grafické zobrazování hry jsem zvolil python knihovnu pygame \cite{pygame}, která, jak si lze z názvu domyslet, slouží k programování jednodušších her v pythonu.
Tato knihovna nabízí kromě různých grafických funkcí, také podporu pro manipulaci s herními objekty. Mimo jiné je v této knihovně zabudovaná podpora pro kontrolu srážek herních objektů.
Mou přirozenou snahou bylo této funkcionality využít, ale toto se později ukázalo být nevhodné.
Prvním problémem bylo, že herní objekty jsou v rámci této knihovny reprezentovány pomocí čtverce, nikoliv jako kruhy a kontrola srážek je tedy realizována jako dotaz, zda se dva odpovídající čtverce pronikají.
Tato kontrola srážky je výpočetně náročnější než kontrola v případě objektů, které jsou reprezentovány kruhem.  
Závažnějším problémem se ukázala být integrace pygame modulu do herního prostředí. Při pokusu o paralelizaci více běhů hry se objevily technické problémy, které se mi ani po značném úsilí nepodařilo vyřešit.
Proto jsem se rozhodl, že pygame modul budu využívat pouze pro grafické vykreslování hry a kontrolu srážek si naimplementuju sám.
V tomto rozsahu pro mě knihovna pygame byla zcela dostačující. Vytvořil jsem si sadu obrázků reprezentující jednotlivé vesmírné objekty a pomocí knihovny pygame je mohu vykreslovat potřebným způsobem.


\section{Hlavní herní cyklus}
Vysvětlili jsme tedy všechny základní prvky, které v této hře potřebujeme a nyní je propojíme dohromady.
K běhu hry potřebujeme inicializovat instanci herního prostředí a dva agenty, kteří budou představovat naše hráče.
a následně můžeme začít v cyklu simulovat hru.
V každém kroku cyklu agenti na zvolí své akce a ty předají zpět hernímu prostředí.
Takto hra běží, dokud herní prostředí neoznámí, že daným krokem hra skončila.
Kostra herní simulace pak vypadá následovně:

\begin{lstlisting}[language=Python]
    env = Enviroment()
    agent_one = Some_agent()
    agent_two = Some_agent()    
    state = env.reset()
    game_over = False

    while not game_over:
        actions_one = agent_one.choose_actions(state)    
        actions_two = agent_two.choose_actions(state)
        
        game_over, state = env.next_step(actions_one, actions_two)
\end{lstlisting}


\section{Instalace a spuštění hry}
Instrukce pro instalaci a spuštění hry jsou popsány v souboru README. (\url{https://github.com/PremekBasta/Asteroids/blob/v0.2/README})

\todo{doplnit vysledný tag}