\chapter{Senzory a akční plány}

\section{Motivace}
Na nejnižší úrovni dostávají agenti od prostředí stav, který obsahuje seznamy všech vesmírných objektů a agenti na to mají reagovat nějakou elementární akcí.
Bylo by proto dobré vymyslet nad detailními informacemi o stavu nějakou informaci vyšší úrovně. A stejně tak by bylo dobré namísto volby nějaké z elemntárních akcí vymyslet způsob vyšší úrovně volby akcí.
A právě tuto abstrakci uskutečníme pomocí senzorů a akčních plánů.
\subsection{Senzor}

Senzorem nazveme metodu, která nám z kompletního stavu reprezentovaného seznamem vesmírných objektů extrahuje nějakou užitečnou informaci, která není ve stavu explicitně zadána.
\section{Akční plán}
Akčním plán chce dosáhnout vyššího cíle. Cílem akčního plánu není žádný elementární důsledek, 

\section{Jednotliv plány}
\subsection{Útok}
\subsection{Sestřelující obrana}
\subsection{Úhybná obrana}
\subsection{Zastavení letu}




