\documentclass[12pt,a4paper]{report}
\setlength\textwidth{145mm}
\setlength\textheight{247mm}
\setlength\oddsidemargin{15mm}
\setlength\evensidemargin{15mm}
\setlength\topmargin{0mm}
\setlength\headsep{0mm}
\setlength\headheight{0mm}
% \openright zařídí, aby následující text začínal na pravé straně knihy
\let\openright=\clearpage

\tolerance=1600

%% Pokud tiskneme oboustranně:
% \documentclass[12pt,a4paper,twoside,openright]{report}
% \setlength\textwidth{145mm}
% \setlength\textheight{247mm}
% \setlength\oddsidemargin{14.2mm}
% \setlength\evensidemargin{0mm}
% \setlength\topmargin{0mm}
% \setlength\headsep{0mm}
% \setlength\headheight{0mm}
% \let\openright=\cleardoublepage

%% Vytváříme PDF/A-2u
\usepackage[a-2u]{pdfx}
%% Přepneme na českou sazbu a fonty Latin Modern
\usepackage[czech]{babel}
\usepackage{lmodern}
\usepackage[T1]{fontenc}
\usepackage{textcomp}

%% Použité kódování znaků: obvykle latin2, cp1250 nebo utf8:
\usepackage[utf8]{inputenc}

%%% Další užitečné balíčky (jsou součástí běžných distribucí LaTeXu)
\usepackage{amsmath}        % rozšíření pro sazbu matematiky
\usepackage{amsfonts}       % matematické fonty
\usepackage{amsthm}         % sazba vět, definic apod.
\usepackage{bbding}         % balíček s nejrůznějšími symboly
			    % (čtverečky, hvězdičky, tužtičky, nůžtičky, ...)
\usepackage{bm}             % tučné symboly (příkaz \bm)
\usepackage{graphicx}       % vkládání obrázků
\usepackage{fancyvrb}       % vylepšené prostředí pro strojové písmo
\usepackage{indentfirst}    % zavede odsazení 1. odstavce kapitoly
\usepackage{natbib}         % zajištuje možnost odkazovat na literaturu
			    % stylem AUTOR (ROK), resp. AUTOR [ČÍSLO]
\usepackage[nottoc]{tocbibind} % zajistí přidání seznamu literatury,
                            % obrázků a tabulek do obsahu
\usepackage{icomma}         % inteligetní čárka v matematickém módu
\usepackage{dcolumn}        % lepší zarovnání sloupců v tabulkách
\usepackage{booktabs}       % lepší vodorovné linky v tabulkách
\usepackage{paralist}       % lepší enumerate a itemize
\usepackage{xcolor}         % barevná sazba

\usepackage{pythontex}
\usepackage{listings}
\usepackage{amsmath}
\usepackage{float}
\newcommand\todo[1]{\textcolor{red}{#1}}
% \usepackage{float}

%%% Údaje o práci

% Název práce v jazyce práce (přesně podle zadání)
\def\NazevPrace{Vesmírná hra s umělou inteligencí}

% Název práce v angličtině
\def\NazevPraceEN{Space game with artificial intelligence}

% Jméno autora
\def\AutorPrace{Přemysl Bašta}

% Rok odevzdání
\def\RokOdevzdani{2020}

% Název katedry nebo ústavu, kde byla práce oficiálně zadána
% (dle Organizační struktury MFF UK, případně plný název pracoviště mimo MFF)
\def\Katedra{Katedra teoretické informatiky a matematické logiky}
\def\KatedraEN{Department of Theoretical Computer Science and Mathematical Logic}


% Jedná se o katedru (department) nebo o ústav (institute)?
\def\TypPracoviste{Katedra}
\def\TypPracovisteEN{Department}

% Vedoucí práce: Jméno a příjmení s~tituly
\def\Vedouci{Mgr. Martin Pilát, Ph.D.}

% Pracoviště vedoucího (opět dle Organizační struktury MFF)
\def\KatedraVedouciho{Katedra teoretické informatiky a matematické logiky}
\def\KatedraVedoucihoEN{Department of Theoretical Computer Science and Mathematical Logic}

% Studijní program a obor
\def\StudijniProgram{Informatika}
\def\StudijniObor{Obecná informatika}

% Nepovinné poděkování (vedoucímu práce, konzultantovi, tomu, kdo
% zapůjčil software, literaturu apod.)
\def\Podekovani{%
Tímto bych chtěl poděkovat svému vedoucímu, který mě v průběhu práce vedl a poskytoval mi podporu vědeckého, praktického i lidského charakteru.
Dále bych chtěl poděkovat svojí rodině, mému bratrovi za projevený zájem o problematiku, kterou jsem se zabýval, mé matce za pomoc s jazykovou stránkou práce a mé přítelkyni za každodenní trpělivou podporu.
}

% Abstrakt (doporučený rozsah cca 80-200 slov; nejedná se o zadání práce)
\def\Abstrakt{%
Součástí této práce je implementace mé vlastní, jednoduché, vesmírné hry, 
která slouží jako experimentální prostředí pro testování různých přístupů umělé inteligence.
Nad stavy a akcemi hry byly vytvořeny abstrakce ve formě senzorických metod a akčních plánů, 
které umožňují jednoduše přecházet z informací nízké úrovně do informací vyšší úrovně a tak pomáhají algoritmům umělé inteligence jednodušeji manipulovat s agenty, kteří se ve hře pohybují.
Jako algoritmy umělé inteligence byly pro hledání inteligentních agentů zvoleny genetické programování a hluboké Q-učení. 
Závěrečná část se soustředí na popsání chování nalezených agentů a vzájemné porovnání výsledků z provedených experimentů.
}

\def\AbstraktEN{%
Part of this thesis consists of the implementation of my own simple space game which serves as an experimenting environment for testing different aproaches of artificial inteligence.
There have been created abstractions in a form of sensoric methods and action plans as a transition between low level and high level information about game state and actions.
These abstractions help algorithms of artifical inteligence with game agent manipulation.
As far as algorithms are considered I chose genetic programming and Deep Q-learning as main aproachces for inteligent agent development.
Final part contains description of behaviour of developed agents and comparison of performed experiments.
}

% 3 až 5 klíčových slov (doporučeno), každé uzavřeno ve složených závorkách
\def\KlicovaSlova{%
{Vesmírná hra} {Umělá inteligence} {Genetické programování} {Hluboké Q-učení}
}

\def\KlicovaSlovaEN{%
{Space game} {Artificial Inteligence} {Genetic programming} {Deep Q-learning}
}

%% Balíček hyperref, kterým jdou vyrábět klikací odkazy v PDF,
%% ale hlavně ho používáme k uložení metadat do PDF (včetně obsahu).
%% Většinu nastavítek přednastaví balíček pdfx.
\hypersetup{unicode}
\hypersetup{breaklinks=true}

%% Definice různých užitečných maker (viz popis uvnitř souboru)
\include{makra}


% \makeatletter
% \setlength{\@fptop}{0pt}
% \makeatother

%% Titulní strana a různé povinné informační strany
\begin{document}
\include{titulka}

%%% Strana s automaticky generovaným obsahem bakalářské práce

\tableofcontents

%%% Jednotlivé kapitoly práce jsou pro přehlednost uloženy v samostatných souborech
\chapter*{Úvod}
\addcontentsline{toc}{chapter}{Úvod}


Počítačové hry mají kromě zábavného prožitku ze samotného hraní další funkci.
Herní prostředí často tvoří samostatný, uzavřený svět se svými vlastními zákonitostmi.
Takovéto prostředí je pro nás vhodné pro tvoření umělé inteligence. 
Agent ve světě dané hry "vidí", v jakém stavu se nachází, má na výběr konečný počet akcí, které může provést
a cíl, kterého chce dosáhnout.  
\chapter{Navržená hra - Asteroidy}

\section{Herní logika}
Jedná se o hru dvou hráčů. 
Každý z hráčů ovládá svou vesmírnou loď.
Prostředí hry má představovat vesmírný prostor, je to ale prostor zjednodušený, proto zde neplatí gravitační ani odporové síly.
To má tedy za následek, že když se vesmírná loď rozletí v nějakém směru, tak v tomto směru letí i nadále i bez dalšího akcelerování.
\par
\label{HraniceProsotru}
Vesmírný prostor je v této hře nekonečný a dalo by se říct jistým způsobem cyklický, pokud vesmírná loď proletí dolní hranicí herního prostoru, tak nezmizí, ani nenabourá, ale objeví se na stejné pozici jen na horní hranici a obráceně. Analogicky to platí i s bočními hranicemi.
Vesmírné lodě sebou mohou proletět a nedojde ke srážce.
Nejsou zde žádné statické překážky, kterým by bylo třeba se vyhnout. Co se ale může srazit s vesmírnou lodí jsou asteroidy.    
\par
Asteroidy vznikají v průběhu hry na náhodných místech a letí náhodným směrem. Nové asteroidy se generují častěji, čím déle hra trvá.
V boji s asteroidy má hráč v zásadě dvě možnosti. Buď se může pokusit danému asteroidu vyhnout, tím že s lodí pohne mimo trajektorii asteroidu, anebo může asteroid sestřelit.
Každý hráč má omezený počet životů a každá srážka lodě s asteroidem ubere hráči část jeho životů.
\par
Hráč má k dispozici dva typy střel, obyčejnou a rozdvojovací. Vystřelená střela má značně vyšší rychlost než vesmírné lodě i než kolem letící asteroidy.
Střely nejsou určeny k přímému zasažení lodě protihráče, vesmírné lodě jsou k nepřátelké střele imunní.
Střely jsou určený k sestřelování letících asteroidů. Asteroidy mohou mít tři velikosti. Náhodně vytvořený asteroid je vždy největší. Každým rozstřelením daného asteroidu vznikají asteroidy o stupeň menší velikosti.
Nově vytvořené asteroidy vznikají na místě původního asteroidu. Asteroid se může rozstřelit různými způsoby. Zde záleží na tom, jakou střelou byl asteroid setřelen. 
\par
V případě střely obyčejné vznikne namísto původního asteroidu jeden menší, který letí stejným směrem jako střela, která ho zasáhla.
V případě střely rozdvojovací se původní asteroid rozstřelí na dva menší, kde každý z nich je oproti směru střely vychýlen o 15\textdegree po a proti směru hodinnových ručiček.
Pokud je zasažen asteroid nejmenší velikosti, tak již žádné další asteroidy nevznikají. 
Rychlost asteroidů je nepřímo závislá na jejich velikosti, čím je asteroid menší, tim vyšší rychlost má.
\par
Asteroidy vzniklé rozstřelením se stávají projektily daného hráče. Hráč nemůže být zasažen asteroidem, který sám vytvořil

\begin{figure}[p]

\includegraphics[width=150mm, height=100mm]{./Obrazky/UkazkaHry.png}
\caption{Screenshot ze hry}
\label{obr01:}
\end{figure}


\section{Cíl hry}
Během hry vzniká postupně více a více asteroidů, čímž je postupně stále obtížnější se všem asteroidům vyhnout, nebo je sestřelit.
Hráč nemůže zranit nepřítele střelou přímo, může se ale snažit rozstřelit nějaký z kolem letících asteroidů tak, aby pomocí nově vzniklých asteroidů trefil nepřítele.
Cílem hráče je ovládat svou loď takovým způsobem, aby vydržel ve hře déle. Hra končí a hráč vítězí, když nepříteli nezbydou žádné další životy.


\chapter{Architektura hry}


\section{Vesmírné objekty}
Všechny vesmírné objekty mají některá data společná. Každý vesmírný objekt má souřadnice své současné polohy a také vektor rychlosti.

\subsection{Asteroidy}
Asteroidy nesou navíc informace o tom, jaké jsou velikosti a zda-li byly vytvořené nějakým z hráčů. 
Na základě těchto dvou informací je asteroidu při vytvoření přiřazen obrázek, pomocí kterého je po dobu své existence vykreslován.

\subsection{Střely}
Vystřelené střely neletí věčně, ale mají omezenou životnost kolik kroků hry budou existovat.
Tato hodnota se nastavuje z konfiguračního souboru z položky \emph{\uppercase{bullet\_life\_count}}.
V každém kroku hry se střele její živostnost sníží o jedna a pokud se dostane na nulu, tak střela bude zničena.
Střele se při vytvoření nastaví úhel, pod kterým poletí. Tento úhel je roven úhlu natočení vesmírné lodi, který měla při vystřelení.
Samozřejmě také u střely musíme evidovat, kterému z hráčů patří, toto je řešeno odkazem na objekt vesmírné lodi, která střelu vystřelila.
Jak již bylo zmíněno v předchozí kapitole, střely jsou dvojího druhu. Příznakem \emph{split} se určuje zda se jedná o střelu obyčenou nebo rozdvojovací


\subsection{Vesmírná loď}
Vesmírná loď má základní polohové informace rozšířené o úhel. Ten se s každou rotací lodě zvětší nebo zmenší o 12\textdegree.
Akcelerace funguje vektorovým sčítáním. K současnému vektoru rychlosti se přičte vektor odpovídající současnému úholu lodi.
Maximální rychlost vesmírné lodi je omezená, v případě že akcelerací vznikne vektor rychlosti, jehož délka je větší než hodnota maximální rychlosti, se směr vektoru zachová, ale požadovaně se zkrátí.

\newpage



\section{Prostředí}

Hra běží v cyklu diskrétních kroků, které dohromady simulují plynulý pohyb hry.
Herní prostředí je insipirováno projektem open ai gym od google
(viz \url{https://gym.openai.com/}). Jedním rozdílem je však přístup k vykreslování hry. V případě \emph{gym.openai} se prostředí vykresluje zavoláním metody \emph{render()} na instanci prostředí zvenku.
Já jsem zvolil přístup jiný. V případě, že chceme hru graficky zobrazovat, předáváme v kontruktoru prostředí grafický modul, který implementuje vykreslování jednotlivých typů vesmírných objektů.
A prostředí už poté objekty graficky vyresluje interně samo. Rozhodnutí, že se má grafický modul volitelně injektovat v konstruktoru a nemá být natvrdo svázen s prostředím, jsem učinil pro větší nezávislost modulů. 
Později se při práci s různými knihovnami pro evoluční algoritmy ukázalo být problematické, že bylo herní prostředí svázáno s grafickým modulem.
\par

Herní prostředí se stará o manipulací všech vesmírných objektů a akcí s nimi spojenými. 
V každém kroku dostává od hráčů akce, které chtějí provést, a prostředí na to odpovídajícím způsobem reaguje. 
Akce každého hráče z hráčů jsou pole, které obsahuje elementární možné akce:
\begin{itemize}
    \item Rotace vlevo
    \item Rotave vpravo
    \item Akcelerace
    \item Obyčejná střela
    \item Rozdvojovací střela
\end{itemize} 
Hráč může provádět více akcí najednou. Na základě přítomných elementárních akcí se provadí dané reakce.
Prostředí se stará o vesmírné objekty přímo. V případě elementárních akcí, které mění rychlost nebo orientaci vesmírné lodi, prostědí zavolá funkce, které požadované změny na vesmírné lodi provede.
A v případě elemntárních akcí střel se na základě polohy a orientace dané vesmírné lodi vytvoří nová střela, kterou opět bude spravovat právě prostředí.


\par


Jedna instance prostředí odpovídá jedné hře. Herní prostředí má dvě základní metody pro řízení hry. 
\newline 
Metoda \emph{reset()} inicializuje hru do počátečního stavu a tento stav vrátí. Tato metoda se musí zavolat před začátkem hry.
\newline 
A druhá metoda \emph{next\_step(actions\_one, actions\_two)}, která na na základě akcí hráčů, převede hru do následného stavu.
V právě této metodě je schovaná celá logika manipulace s vesmírnými objekty.

\newpage
\begin{lstlisting}[language=Python]
def next_step(self, actions_one, actions_two):
    self.step_count = self.step_count + 1
    self.reward_one = 0
    self.reward_two = 0

    self.handle_actions(actions_one, actions_two)
    self.generate_asteroid()
    self.check_collisions()
    self.move_objects()
    if self.draw_modul is not None:
        self.render()

    (game_over, player_one_won) = self.check_end()
    if not game_over:
        self.reward_one += 1
        self.reward_two += 1

    current_state = State(self.asteroids_neutral, 
                          self.rocket_one, 
                          self.asteroids_one, 
                          self.bullets_one,
                          self.rocket_two, 
                          self.asteroids_two, 
                          self.bullets_two)

    return self.step_count, \
           (game_over, player_one_won), \
           current_state, \
           (self.reward_one, self.reward_two)
\end{lstlisting}
\newpage



\subsection{Stav prostředí}
Úplná informace o všech objektech v prostředí.

\section{Agent}
Agent se na základě informace o současném stavu prostředí rozhodne o své následné akci. 


\section{Grafické prostředí}

Mezi nejvíce citované statistické články patří práce Kaplana a~Meiera a~Coxe
\citep{KaplanMeier58, Cox72}. \citet{Student08} napsal článek o~t-testu.

projektu ACCEPT jsou uvedeny v~práci \citet*{Genberget08}.

\chapter{Senzory a akční plány}

\section{Motivace}
Na nejnižší úrovni dostávají agenti od prostředí stav, který obsahuje seznamy všech vesmírných objektů a agenti na to mají reagovat nějakou elementární akcí.
Bylo by proto dobré vymyslet nad detailními informacemi o stavu nějakou informaci vyšší úrovně. A stejně tak by bylo dobré namísto volby nějaké z elemntárních akcí vymyslet způsob vyšší úrovně volby akcí.
A právě tuto abstrakci uskutečníme pomocí senzorů a akčních plánů.
\subsection{Senzor}

Senzorem nazveme metodu, která nám z kompletního stavu reprezentovaného seznamem vesmírných objektů extrahuje nějakou užitečnou informaci, která není ve stavu explicitně zadána.
\section{Akční plán}
Akčním plán chce dosáhnout vyššího cíle. Cílem akčního plánu není žádný elementární důsledek, 

\section{Jednotliv plány}
\subsection{Útok}
\subsection{Sestřelující obrana}
\subsection{Úhybná obrana}
\subsection{Zastavení letu}





\chapter{Algoritmy umělé inteligence}
V této kapitole se seznámíme se dvěma přístupy, které spadají do odvětví umělé inteligence. 
Představíme jejich základní myšlenku. Zmíníme, kde se tyto algoritmy dají použít a v následující kapitole je i využijeme v našem herním prostředí pro trénování inteligentních agentů.

\section{Genetické programování}

\subsection{Základní princip}
\cite{fieldguide}
Genetické programování je evoluční technika, která vytváří počítačové programy. Cílem genetického programování je vyřešit co nejlépe zadaný problém, neklademe však žádné požadavky na to, jakým způsobem je potřeba ho vyřešit.
\par
Genetické programování spadá do kategorie evolučních algoritmů, s těmi se pojí jistá terminilogie.
Každý program budeme nazývat jedincem a jejich množinu populací. Algoritmus poběží iterativně v generacích. 
V každé iteraci se provede výběr některých jedinců z populace, ti se pomocí genetických operátorů a případných mutací upraví, a na závěr se rozhodne, kteří z nich přežijí do další generace.
Jedince, kteří se v generaci vysktují na začátku iterace, nazývame rodiče. A podobně jedince, kteří byli na konci iterace vybráni do další generace, se nazývají potomci. 
Každý program představuje jedno konkrétní řešení daného problému. Kvalitu tohoto řešení ohodnocujeme takzvanou fitness funkcí. Čím vyšší hodnotu tato funkce jedinci přidělí, tím je lepším řešením daného problému.


\subsubsection{Reprezentace}
Programy jsou v genetickém programování obvykle reprezentované syntaktickými stromy. Stromy ve vnitřních uzlech obsahují funkce (neterminály) a v listech terminály.
Vyhodnocení stromu následně probíhá od listů ke kořeni a výsledek programu je roven hodnotě v kořeni.


\subsubsection{Inicializace populace}
Na začátku evoluce potřebujeme inicializovat populaci, ta je na počátku tvořena náhodně vytvořenými jedinci. K vytváření náhodných jedinců se používá kombinace dvou přístupů.
Prvním z nich je vytváření jedinců s pevně danou hloubkou stromu, kde v listech jsou vždy pouze terminály. V druhém stavění stromů probíhá náhodně z předem určeného počtu neterminálů.
Po vyčerpání počtu neterminálů se opět pouze doplní terminály jako listy. Tato metoda vytváří jedince různých velikostí a tvarů.
Častou praxí je počáteční populaci vytvořit tak, že každá z metod vytvoří polovinu jedinců. 


\subsubsection{Selekce}
V každé iteraci chceme vybrat několik jedinců, nad kterými budeme provádět různé genetické úpravy. Snahou selekce je volit jedince s vyšší hodnotou fitness funkce.
Asi nejpoužívanější metodou selekce je turnajová selekce. V té se náhodně zvolí daný počet jedinců, kteří se mezi sebou porovnají na základě jejich hodnoty fitness funkce a nejlepší z nich je poté zvolen dso výběru.
Dalším z mnoha metod (\cite{selekcniMetody}) selekce je ruletová selekce. Zde je pravděpodobnost zvolení jedince do výběru přímo úměrná jeho hodnotě fitness funkce.
Pravděpodobnost výběru i-tého jedince je 
\[p(i) = \frac{f(i)}{\sum_{j=1}^{n} f(j)}. \]
Ruletová selekce má jednoduchou implementaci a zároveň je rychlá na výpočet. 
Jejím problémem je ale předčasná konvergence k lokálnímu optimu.
\par
Genetické operátory mohou občas upravit nejlepšího jedince tak, že se jeho hodnota fitness funkce výrazně zhorší.
Z tohoto důvodu se často používá technika zvaná elitismus, která automaticky do další generace vybere pár nejlepších současných jedinců. 
Tímto způsobem máme garantováno, že neztracíme nejlepší současné jedince.

\url{https://www.researchgate.net/publication/259461147_Selection_Methods_for_Genetic_Algorithms}


\subsubsection{Genetické operátory}
Genetické operátory jsou dvojího druhu, křížení a mutace. Myšlenkou křížení je ze dvou jedinců, nazývejme je rodiče, vytvořit nového potomka, který bude tvořen kombinací obou z jeho rodičů.
V genetickém programování pracujeme s jedinci reprezentovanými stromy, proto křížení jedinců je realizováno křížením jejich stromů. V každém z rodičů se pro křížení zvolí jeden uzel stejného typu. 
Výsledný potomek má strukturu podobnou prvnímu rodiči, jen na původním místě vybraného uzlu bude nyní podstrom, který je zavěšený pod vybraným uzlem druhého rodiče.
\par
Druhým genetickým operátorem je mutace, ta již nepotřebuje mít dva rodiče, ale úprava se provede nad jedincem samotným.
Mutovat můžeme v jedinci buď jediný bod, nebo celý podstrom. V případě mutace podstromu se namísto vybraného podstromu vygeneruje zcela nový podstrom. 
Toto v zásadě představuje křížení s novým náhodně vytvořeným jedincem.
\newline
Mutace jediného bodu změní náhodně jediný uzel ve stromě. V případě terminálu se může vybrat libovolný jiný terminál. A v případě vnitřního uzlu může být vybraný libovolný jiný neterminál, který je stejného typu.


\subsubsection{Silná a volná typovanost}
Ve volně typovaném genetickém programování nezadáváme typy funkcí ani terminálů. Jediné co musíme u funkcí určit je jejich arita. Na základě arity se následně generují a mutují jedinci korektně.
Obvykle se nám ale více bude hodit silně typované genetické programování, zde určujeme u všech funkcí nejen jejich aritu, ale také typ každého z argumentů dané funkce a také typ návratové hodnoty.
Podobně musíme určit i hodnotové typy terminálů.
Na závěr určíme hodnotové typy celého problému a algoritmus už se postará o to, aby byly typové podmínky pro všechny jedince dodrženy.
\url{https://deap.readthedocs.io/en/master/tutorials/advanced/gp.html}


\subsection{Využití}
Genetické programování je využitelné ve všech problémech, kde jsme schopní vymyslet způsob, jak jedince představujícího řešení problému reprezentovat a jak počítat fitness funkci, která dané řešení ohodnotí.
To tedy znamená, že možnosti využití jsou téměř nekonečné. 

\todo{volně přeloženo a zkráceno z původního textu str. 126}
Obecně se genetické programování ukazuje být vhodné ve všech problémech, které splňují některou, z následujících podmínek:

\begin{itemize}
    \item Vzájemné vztahy zkoumaných proměnných nejsou dobře známy, nebo je podezření, že jejich současné porozumění může být mylné.
    \item Nalezení velikosti a tvaru hledaného řešení je částí řešeného problému.
    \item Existují simulátory pro testování vhodnosti zadaného řešení, ale neexistují metody pro přímé získání dobrých řešení.
    \item Obvyklé metody matematické analýzy nedávají, nebo ani nemohou být použity pro získání analytického řešení.
    \item Přibližné řešení je zcela postačující.
    
\end{itemize}



\subsubsection{Symbolická regrese}
V mnoha problémech je naším cílem nalézt funkci, jejíž hodnota splňuje nějakou požadovanou vlastnost. Toto známe pod názvem symbolická regrese.
Obyčejná regrese má obvykle za cíl nalézt koeficienty předem zadané funkce, tak aby co nejlépe odpovídala daným datům.
Zde je problém, že pokud potřebná funkce nemá stejnou strukturu, jako zadaná funkce, tak dobré koeficienty nenalezneme nikdy a musíme zkusit hledat funkci jiné struktury.
Tento problém může vyřešit právě symbolická regrese. Ta hledá vhodnou funkci, aniž by na začátku měla očekávání o její struktuře.
Uvedeme triviální příklad. Řekněme, že hledáme výraz, jehož hodnoty odpovídají polynomu $x^2+x+1$ na intervalu $[-1,1]$.
Budeme hledat funkci jedné proměnné $x$, proto $x$ přidáme jako terminál. Dále přidáme jako terminály číselné konstanty (například -1,1,2,5,10), které budou sloužit pro hledání koeficientů.
Pro náš případ bude stačit, když si jako aritmetické funkce přidáme ty základní, tedy sčítání, odečítání, násobení a dělení.
Fitness funkci můžeme zvolit jako součet absolutních hodnot rozdílů výrazu jedince a hledaného výrazu $x^2+x+1$.
Toto stačí pro spuštění evolučního algoritmu. 
V takto triviálním případě se hledané řešení nalezne pravděpodobně vezmi brzo a bude mít jednu z podobu stromu reprezentujícího daný výraz (viz strom výrazu \ref{obr04:GrafFormule})


\begin{figure}[p]\centering
\includegraphics[width=70mm, height=70mm]{./Obrazky/formule_graph_2.png}
\caption{Strom výrazu}
\label{obr04:GrafFormule}
\end{figure}



\newpage
\section{Hluboké Q-učení}
Než se dostaneme k samotnému fungování algoritmu hlubokého Q-učení, musíme nejprve vybudovat jeho stavební kameny.

% \subsection{Základní principy}
\subsection{Neuronová síť}
První důležitý koncept, který v rámci hlubokého Q-učení budeme potřebovat, jsou neuronové sítě. 
I zde se nejprve budeme muset seznámit se základní jednotkou - perceptronem, než budeme schopni říct, co neuronové sítě jsou.
\par
Perceptron je algoritmus, který má na vstupu několik hodnot $x_i$ a jeden výstup. S každou vstupní hodnotou je spojena jedna váha $w_i$.
Kromě vah vstupů obsahuje perceptron ještě tzv. práh. Perceptron spočítá vážený součet vstupů a porovná výsledek s prahem. Pokud je výsledek větší než práh, tak perceptron vrací hodnotu 1, jinak 0.
Můžeme zde ale využít triku pro zbavení se prahu. K prahu můžeme přistupovat jako k další váze s konstantním vstupem -1.
V takovém případě pak perceptron provádí porovnání váženého součtu vstupů s hodnotou 0. 
\newline
Matematicky zapsáno:
\[f(\sum_{i=0}^{n} w_if_i)\], kde $f$ vrací 1 pro $x>0$ a 0 jinak.

Trénování daného perceptronu probíhá pomocí předkládání dvojic vstupů a výstupů $(x,y)$ z trénovací množiny a upravování vah následujícím způsobem:
\newline
\[w_i = w_i + r(y-f(x_i)x_i\], kde $r$ je parametr učení.

Rozhodovací hranice perceptronu představuje nadrovinu ve vstupním prostoru. Lze ukázat, že trénování perceptronu zkonverguje, pokud jsou třídy v datech lineárně separabilní.

Většinou ale řešíme problémy, kde třídy v datech lineárně separabilní nejsou, v takových případech nám jeden perceptron nestačí a potřebujeme jich využít víc.
Spojením více perceptronů získáváme dopřednou neuronovou síť.
Ta se skládá z vrstev perceptronů, kde vstupy perceptronů první vrstvy jsou samotná data $x$ a vstupy perceptronů v dalších vrstvách jsou rovny výstupům perceptronů z vrstvy předchozí.

Pro trénování více vrstvých perceptronů se používá gradientní metoda. Z toho důvodů se využívají jiné funkce $f$, než ta zmíněná nahoře, ta má totiž gradient ve většině případů roven 0.
Typickým příkladem používané funkce je sigmoid \[f(x) = \frac{1}{1+e^{-x}}.\]

Následně musíme zvolit chybovou funkci $L(x,y|w)$ neuronové sítě. Zde se typicky využívá střední kvadratická chyba (ang. Mean squared error).
Chybová funkce se derivuje podle vah v síti a následně se jednotlivé váhy upraví.
\newline
\[w_i = w_i + \alpha\frac{\partial L(x,y|w)}{\partial w_i}\]

\url{https://ocw.mit.edu/courses/health-sciences-and-technology/hst-947-medical-artificial-intelligence-spring-2005/lecture-notes/ch7_mach3.pdf}
\todo{strana 9 až 13}


\subsection{Zpětnovazební učení}
Podobně jako v předchozí sekci, také zde musíme nejprve zavést nové pojemy a terminologii.
Popišme si nejprve, co je to markovský rozhodovací prostor.
Markovský rozhodovací prostor popisuje prostředí a je definován čtveřicí $(S,A,P,R)$, kde $S$ je množina stavů, $A$ je množina všech akcí (případně $A_s$ představuje množinu akcí, které mohou být provedeny ve stavu $s$), 
$P: S \times A \times S \rightarrow [0,1]$ představuje přechodovou funkci, 
kde $P_a(s,s')$ vrací pravděpodobnost, že se aplikováním akce $a$ ve stavu $s$ dostaneme do stavu $s'$, 
a $R: S \times A \times S \rightarrow \mathbb{R}$ představuje funkci odměn $R_a(s,s')$, která vrací odměnu, kterou agent obdrží, pokud ve stavu $s$ provede akci $a$ a dostane se tak do stavu $s'$.
Přechodová funkce i funkce odměn musí navíc splňovat podmínku, že musí být jejich hodnoty nezávislé na předchozích stavech.

\url{https://ocw.mit.edu/courses/aeronautics-and-astronautics/16-410-principles-of-autonomy-and-decision-making-fall-2010/lecture-notes/MIT16_410F10_lec22.pdf}

Chování agenta v prostředí můžeme popsat pomocí strategie $\pi: S \times A \rightarrow [0,1]$, ta určuje pravděpodobnost, že se agent ve stavu $s$ rozhodne pro akci $a$.
Pro agenta v prostředí ještě definujme jeho celkovou odměnu jako \[\sum_{t=0}^{\infty} \gamma^tR_{a_t}(s_t,s_{t+1})\], kde $\gamma<1$ je diskontní faktor, díky kterému je suma konečná a $a_t$ je akce agenta vybraná v kroku t.
Cílem zpětnovazebního učení je nalézt optimální strategii $\pi^\star$, kde $a_t=\pi^\star(s_t)$, takovou, že její celková odměna je maximální.

Hodnotu stavu $s$ při použití strategie $\pi$ lze definovat jako 
\newline
\[V^{\pi}(s)=\mathbf{E}[\sum_{t=o}^{\infty} \gamma^tr_t|s_0=s]\], kde $r_t$ značí odměnu získanou v kroku t.
Podobně také můžeme definovat hodnotu $Q^{\pi}(s,a)$ akce $a$ provedené ve stavu $s$ při následování strategie $\pi$.
\[Q^\pi(s,a)=\sum_{s'}P_a(s,s')[R_a(s,s') + \gamma\sum_{a'} \pi(s',a')Q^\pi(s',a')]\]
Z Bellmanovy rovnice pro optimální strategie platí:
\[Q^*(s,a)=R_a(s,s') + \gamma\max_{a'}Q_k(s',a')\]

Agent ke zlepšování své strategie může využívat přechodové funkce a funkce odměn. Často ale agent hodnoty jednotlivých stavů předem nezná a musí se je učit za běhu.
Zároveň musí volit mezi explorací tj. prohledáváním prostoru a exploatací tj. využíváním známého. Zde využijeme $\epsilon$-hladového (ang. $\epsilon$-greedy) přístupu, kdy s pravděpodobností $\epsilon$ vybere náhodou akci a s pravděpodobností $1-\epsilon$ vybere nejlepší známou akci.

Nyní se už dostáváme ke Q-učení. $Q$ je reprezentována jako matice zpočátku inicializovaná samými nulami. Agent následně ve stavu $s_t$ vybírá například $\epsilon$-hladovým přístupem akci $a_t$, získá od prostředí odměnu $r_t$ a přesune se do stavu $s'$.
Na základě těchto informací se provede aktualizace matice následovně:
\newline
\[Q(s_t,a_t) \leftarrow (1-\alpha)Q(s_t,a_t) + \alpha(r_t + \gamma(\max_a(Q(s_{t+1},a))))\]




\subsection{Hluboké Q-učení}
Po seznámí se se základy neuronových sítí a zpětnovazebního učení můžeme přejít k samotnému hlubokému Q-učení.
Hluboké Q-učení následuje stejnou myšlenku jako obyčejné Q-učení, také chceme nalézt odměnu vybrané akce v současném stavu.
Hlavním rozdílem oproti normálnímu Q-učení je způsob, jak se Q hodnoty reprezentují. V normálním Q-učení jsou Q hodnoty uložené v matici, ta může být v případě velkých prostorů příliš obrovská a Q-učení pak může probíhat velmi pomalu, nebo dokonce vůbec.
V Hlubokém Q-učení bude Q reprezentováno pomocí neuronové sítě.

Trénování se provádí pomocí porovnávání rozdílu aktuální odměny $R_a(s,s')$ prostředí od odměny spočítané pomocí Bellmanovy rovnice z Q.
Cílem je tedy minimalizovat rozdíl mezi \[Q(s,a) \hspace{5mm}\text{a}\hspace{5mm}  R_a(s,s') + \gamma\max_{a'}Q_{\theta}(s',a')\], kde $Q_{\theta}$ jsou parametry neuronové sítě reprezentující matici Q.
Chybovou funkcí pro trénování neuronové sítě pak může být střední čtvercová chyba tohoto rozdílu.



\subsection{Využití}
V praxi bylo hluboké Q-učení použito například pro naučení se hraní atari her. 
\url{https://www.cs.toronto.edu/~vmnih/docs/dqn.pdf}

Namísto ručně zpracovaných informacích o stavu hry zde byly jako vstupy využity přímo vizuální výstupy hry.
Q-síť je proto reprezentována konvoluční neuronovou sítí, která na vstupu bere vektor pixelů obrázku hry a na výstupu vrací odhad budoucích odměn pro každou z možných akcí.
Modelu nebylo řečeno nic o principu fungování hry. Model se učil pouze na základě vizuálního výstupu, odměn, které dostával od prostředí, konečných stavů a množiny možných akcí, tedy podobně, jak by se hru učil hrát člověk.

Stejná neuronová síť byla použitá na sedmi různých atari hrách. Na šesti z nich překonala všechny stávající přístupy, které využívaly algoritmy zpětnovazebního učení a na třech z nich překonala výsledky nejlepších lidských hráčů.



\tolerance=1600
\chapter{Provedené experimenty}
V předchozí kapitole jsme se seznámili s algoritmy, které jsou aplikovatelné i v našem prostředí.
V této kapitole se pomocí různých experimentů pokusíme nalézt agenty, kteří se budou v jistém pojetí chovat inteligentně.

Abychom mohli výsledky experimentů vyhodnocovat, případně vzájemně mezi sebou porovnávat, potřebujeme mít na výsledky konkrétní kritéria.
Při testování obranného akčního plánu se ukázalo, že agent, který využívá pouze obranného akčního plánu k sestřelení nejbližšího asteroidu, se kterým vesmírné lodi hrozí srážka, dokázal díky dobré obraně zdatně přežívat ve hře mnoho kroků.
A zároveň je tento obranný agent vzhledem k využívání pouze jednoho akčního plánu dostatečně neinteligentní na to, abychom ho mohli použít jako referenčního agenta.

Jako kritérium pro hodnocení výsledků experimentů tedy bude sloužit souboj mezi výsledným agentem experimentu a obranným agentem.
První z nich, kterému se podaří zvítězit desetkrát, bude označen za vítěze. Výsledky těchto soubojů pak budou sloužit jako porovnání mezi jednotlivými provedenými experimenty.

Kromě primárního cíle nalézt agenty s inteligentním chováním máme i cíl sekundární, a to nalézt agenty, kteří se budou chovat zajímavě ve smyslu pestrosti akcí, akčních plánů, nebo také zajímavě v tom smyslu, že bude jejich chování působit jako chování lidského hráče.

\section{Genetické programování}
V prvních experimentech využijeme již zmíněného genetického programování. Nezbytným požadavkem pro jeho využití je existence reprezentace jedince a fitness funkce, která ho ohodnotí. 
Obojí dokážeme jednoduše vyřešit.
Jedinec bude představovat rozhodovací funkci, která se na základě vstupních argumentů rozhodne, který akční plán bude vybrán.

V našem případě máme hru, kde spolu dva hráči soupeří a hra končí výhrou jednoho z hráčů.
Přesně tohoto můžeme ve fitness funkci jedince využít. 
Pro využití výsledku hry, jak jedinec ve hře dopadl, musíme nejprve zvolit proti jakému hráči bude jedinec, který je předmětem našeho zájmu, hrát.


Pro experimentování s genetickým programováním jsem zvolil knihovnu deap pro python. Zde lze jednoduše konfigurovat evoluční algoritmus na konkrétní řešený problém.
Stačí popsat jak reprezentovat jedince a jak se vypočítá jeho fitness funkce a zbytek knihovna vyřeší za nás.

\subsubsection{Reprezentace jedince}

Ve 3. kapitole jsme si vybudovali abstrakce v podobě senzorů a akčních plánů a těch zde budeme chtít využít.
Jedince budeme podobně jako u symbolické regrese reprezentovat stromem.
Strom jedince budeme budovat prvky z následující množiny terminálů a neterminálů.

\begin{itemize}
\item{
    Terminály:
    \begin{itemize}
        \item Vstupní argumenty rozhodovací funkce (viz níže)
        \item Celočíselné konstanty -1, 1, 3, 5, 10, 100
        \item Nulární funkce vracející hodnoty reprezentující zvolený akční plán  
    \end{itemize}    
    }    
\end{itemize}
Jako argumenty funkce jsem zvolil následující hodnoty: délky všech čtyř akčních plánů a počet kroků před srážkou vesmírné lodi s asteroidem.
    Délky akčních plánů se pohybují v intervalu $(1,100)$, proto jsou číselné konstanty zvoleny tak, aby se jejich sčítáním a násobením lehce dosáhlo dalších hodnot z tohoto intervalu.


\begin{itemize}

\item{
    Neterminály:
    \begin{itemize}
        \item Aritmetické operace sčítání a násobení
        \item Funkce \emph{compare}
        \item Funkce \emph{if\_then\_else}
    \end{itemize}
}
\end{itemize}
Z aritmetických operací nám stačí sčítání a násobení. Operaci odčítání získáme pomocí sčítání a násobení konstantou -1. 
Hodnoty z intervalu $(1,100)$ jednoduše získáme také pomocí sčítáním a násobením potřebných konstant, proto pro operaci dělení není důvod.
Všechny aritmetické operace jsou typu $(int, int) \rightarrow int$
Funkce \emph{compare} je typu $(int, int) \rightarrow bool$, ta vrací zda je první argument větší než druhý.
Poslední použitá funkce \emph{if\_then\_else} je typu 
\newline
$(bool, ActionPlanEnum, ActionPlanEnum)\rightarrow ActionPlanEnum$. 
Tato funkce dostává jako argumenty výraz typu bool a následně dvě hodnoty reprezentující akční plány. 
Na základě pravdivosti výrazu vrací funkce první nebo druhou z hodnot akčních plánů.

Takto popsaná reprezentace jedince bude použita ve všech následujících experimentech. To, v čem se budou experimenty lišit, je způsob výpočtu fitness funkce a průběh evolučního algoritmu.

\subsection{Experiment 1: Soupeření s obranným agentem}
Cílem tohoto experimentu bylo vyvinout agenta, který bude lepší než obranný agent.
Hra je pokaždé velmi náhodná, tedy zahrání jedné hry by mělo nízkou vypovídající hodnotu. Proto jsem pro přesnější informaci zvolil zahrání šesti her.
Hodnota zahrané hry se skládá z více částí.
\begin{itemize}
    \item Počet kroků trvání hry
        \newline
        Myšlenkou je zde, obzvláště v počátku evoluce, upřednostňovat takové jedince, kteří dokáží vydržet ve hře co nejdéle, tedy nejsou ve hře okamžitě poraženi.
        Pro představu, délky her se pohybují přibližně v intervalu $(900, 2900)$ kroků.
    \item Penalizace za nevyužití některého z plánů
        \newline
        Během hry se udržuje historie, kolikrát se agent rozhodl pro každý z akčních plánů.
        Za každý ze čtyř akčních plánů, který agent ani jednou během hry nezvolil bude přičtena penalizace -500. Cílem těchto penalizací je upřednostňovat takové jedince, kteří používají všechny akční plány a tím pádem mají pestřejší chování. 
    \item Bonus (penalizace) za výhru (prohru)
        \newline
        Toto je asi nejdůležitější část. Pro zdůraznění rozdílu mezi vyhranými a prohranými hrami se v případě výhry přičte k výsledku hodnota 2000 a v případě prohry se 2000 odečte.
        Motivací mohou být následující dvě situace. Řekněme, že v jedné hře se podařilo jedinci dlouho bránit a dokázal vydržet 2500 kroků hry a poté prohrál. A v další situaci jedinec porazil soupeře v rychlých 1200 krocích. 
        Bez bonusu za vyhranou hru, by prohraná hra získala jedinci daleko vyšší hodnotu, než situace z druhé hry, kterou vyhrál.        
    
\end{itemize}

Algoritmus byl spuštěn s následujícími parametry:
\begin{itemize}
    \item Velikost populace: 30
    \item Pravděpodobnost křížení: 60\%
    \item Pravděpodobnost mutace: 20\%
    \item Genetické operátory: křížení dvou rodičů, jednobodová mutace a mutace celého podstromu
    \item Počet generací: 100
    \item Metoda selekce: turnajová selekce
\end{itemize}

Výsledný nejlepší jedinec bohužel nesplnil naše očekávání a nedokázal obranného agenta porazit. 
V souboji jedinec nejen nedokázal konkurovat obrannému agentovi, ale ani se neřídil příliš pestrou strategií.

V 95\% volil, stejně jako obranný agent, obranný akční plán a ve zbylých pár procentech volil všechny zbylé akční plány (viz \ref{Výsledek experimentu 01}).
Vyžívání všech akčních plánů bylo pravděpodobně dosaženo právě skrze vysokou penalizaci při nepoužití libovolného z nich, ale vidíme, že agent je volil spíše právě z tohoto důvodu, než že by je chtěl aktivně využívat v rámci své strategie.

\begin{figure}[H]\centering
\includegraphics[width=125mm, height=100mm]{./Obrazky/Experiment01Results.png}
\caption{Výsledek experimentu 1}
\label{Výsledek experimentu 01}
\end{figure}


\subsection{Experiment 2: Postupné zaměňování úspěšnějšího jedince}
V tomto experimentu nebylo cílem porazit konkrétního, stálého agenta jako v předchozím případě.
Zde bylo cílem postupně vybudovat nejzdatnějšího jedince.
Stejně jako v předchozím experimentu i zde fitness funkce spočívá v zahrání šesti her,
avšak zde nebudeme hrám přiřazovat žádnou číselnou hodnotu, ale spokojíme se s jednoduchou informací, který z agentů v dané hře zvítězil.
\par
V průběhu evoluce si budeme pamatovat současného nejlepšího jedince. 
Na začátku bude tento jedinec vybrán zcela náhodně. Obvyklým způsobem vytvoříme počáteční populaci a započneme evoluci.
Fitness funkce jedince bude počítat poměr, kolik ze šesti zahraných her jedinec vyhrál v souboji se současně nejlepším nalezeným jedincem.
Evoluce hledá řešení, která budou proti současnému nejlepšímu jedinci co nejúspěšnější.
V každé 3. generaci se následně kontroluje, zda již náhodou nebyl v populaci nalezen jedinec, který, pro současnou situaci, nejlepšího jedince porazil alespoň v pěti ze šesti her.
Pokud ano, tak takový jedinec bude nově zvolen jako nejlepší a evoluce bude pokračovat stejným způsobem dál.
\par
Po výměně nejlepšího jedince musíme nově přepočítat fitness funkci všech stávajících jedinců v populaci, protože jejich současná hodnota se vztahovala k původnímu soupeři.
Rovněž musíme, ze stejného důvodu, smazat všechny jedince ze síně slávy (ang. Hall of fame), kde se průběžně ukládají nejlepší jedinci spolu s hodnotou jejich fitness funkce.

\par
Všechny tyto změny už nelze v knihovně deap nakonfigurovat přímočarým způsobem jako v předchozím experimentu, ale bylo zapotřebí upravit samotnou kostru evolučního algoritmu.

\par
Algoritmus byl spuštěn s následujícími parametry
\begin{itemize}
    \item Velikost populace: 10
    \item Pravděpodobnost křížení: 60\%
    \item Pravděpodobnost mutace: 20\%
    \item Genetické operátory: křížení dvou rodičů, jednobodová mutace a mutace celého podstromu
    \item Počet generací: 450
    \item Metoda selekce: turnajová selekce
\end{itemize}

Výsledného agenta jsme nechali zahrát souboj s obranným agentem a tentokrát přinesl experiment daleko lepší výsledky.
Nalezený agent se oproti předchozímu experimentu dokázal naučit lépe útočit, volil útočný akční plán téměř ve 40\% případech a díky tomu dosáhl našeho primárního cíle. V Souboji porazil obranného agenta se skóre 10:0 a tím splnil primární cíl porazit obranného agenta.
Nicméně ani tentokrát se agent nenaučil nic jiného než obranu a útok (viz \ref{Výsledek experimentu 02}) a proto agentovo chování opět není příliš pestré. 

 


\begin{figure}[H]\centering
\includegraphics[width=125mm, height=100mm]{./Obrazky/Experiment02Results.png}
\caption{Výsledek experimentu 2}
\label{Výsledek experimentu 02}
\end{figure}



\newpage
\subsection{Experiment 3: Postupné zaměňování úspěšnějšího jedince bez obranného akčního plánu}

V předchozích experimentech se nám v obou případech podařilo vytvořit agenty, kteří v drtivé většině stavů rozhodují jen mezi obranným a útočným plánem.
To má za následek, že se agenti po celou dobu hry pouze otáčejí a střílejí, ale zůstavají při tom na jednom stejném místě.
V tomto experimentu tomuto problému zkusíme předejít, tím, že donutíme agenta bránit se uhýbáním namísto sestřelování nebezpečných asteroidů.

\par
\tolerance=10000
Experiment probíhá stejným způsobem jako v předchozím případě, jen s tím rozdílem, že agentovi zakážeme používání obranného plánu. 
Z argumentů rozhodovací funkce odstraníme informaci o obranném plánu. 
A z množiny terminálů používaných při tvorbě programů odstraníme nulární funkci reprezentující obranný akční plán.

\par
\tolerance=1600
To je vše co je potřeba změnit a zbylá logika může zůstat stejná jako v předešlém experimentu.

Algoritmus byl spuštěn s následujícími parametry
\begin{itemize}
    \item Velikost populace: 10
    \item Pravděpodobnost křížení: 60\%
    \item Pravděpodobnost mutace: 20\%
    \item Genetické operátory: křížení dvou rodičů, jednobodová mutace a mutace celého podstromu
    \item Počet generací: 2000
    \item Metoda selekce: turnajová selekce
\end{itemize}

S výsledným agentem jsme opět provedli souboj s obranným agentem.
První čeho si můžeme všimnout je, že náš agent v souboji prohrál s výsledkem 2:10, tedy bez obranného akčního plánu nebyl schopný tak úspěšně konkurovat obrannému agentovi.
Druhá skutečnost, která stojí za povšimnutí je průměrná délka hry, ta byla v průměru přibližně o 500 kroků kratší než v předchozím experimentu. 
Z toho vyplývá, že využívání úhybného akčního plánu k přežívání není tak účinné, jako bránění se pomocí obranného akčního plánu. 
To ale není nijak překvapivé. Pro agenta je prostředí tím víc nebezepečné, čím více je v něm nebezpečných asteroidů. 
Používání úhybného akčního plánu vede k vyhnutí vesmírné lodi před nebezepčným asteroidem, ne před jeho zničením, jako je to u obranného akčního plánu. To má za následek, že v případě úhybného akčního plánu agent neredukuje počet nebezepčných asteroidů a mnohem dříve se dostane do stavu, kdy je pro agenta příliš obtížné se roji asteroidů vyhnout. 

\par
Zajímavým výsledkem experimentu je také to, že, přestože agent používá v rámci úhybného akčního plánu akceleraci pro obranu velmi často, ani tentokrát nepoužívá zastavovací akční plán (viz \ref{Výsledek experimentu 03}).
To ukazuje, že zastavování letu není pro získání lepších výsledků stěžejní.
\par
Výsledný agent v souboji jednoznačně dosáhl špatných výsledků, ale pokud se na hru podíváme z lidského pohledu, tak oproti agentům, kteří zůstávají po celou dobu hry na místě, působí agentovo chování mnohem zajímavěji.


\begin{figure}[H]\centering
\includegraphics[width=125mm, height=100mm]{./Obrazky/Experiment03Results.png}
\caption{Výsledek experimentu 3}
\label{Výsledek experimentu 03}
\end{figure}




\newpage
\section{Hluboké Q-učení}
Herní prostředí nám v každém kroku vrací odměnu, kterou oba z hráčů za svou akci obdrželi. Tuto informaci jsme v experimentech provedených v rámci genetického programování nevyužili, ale zde bude mít zásadní roli.
Za každý krok, kdy hra ještě neskončila, získávají agenti automaticky odměnu 1. Na konci hry agent obdrží vysokou odměnu 2000 v případě výhry a v případě prohry naopak získá penalizaci v podobě odměny vysoké záporné hodnoty -1000.
Odměnu za výhru, nebo prohru získá agent až na úplném konci hry, to může ztěžovat učící proces. Proto prostředí dává agentům i průběžné menší odměny, pro lepší možnost učení se.

Konkrétně to jsou následující odměny:
\begin{itemize}
    \item Sestřelení asteroidu
        \newline
        Za každý sestřelený asteroid získává agent odměnu hodnoty 5.
    \item Zasažení nepřátelské vesmírné lodi asteroidem 
        \newline
        Zranění nepřítele je právě to, co agent potřebuje pro přiblížění se vítězství, proto za každé takové zasažení získává od prostředí odměnu v hodnotě 20.
    \item Zasažení vlastní vesmírné lodi asteroidem
        \newline
        Takový stav je pro agenta znevýhodňující a cílem je se mu vyvarovat, proto za takovýto stav agent od prostředí dostává penalizaci v hodnotě -10.
\end{itemize}


V rámci učení agentů budeme využívat $\epsilon$-hladového (ang. $\epsilon$-greedy) přístupu.
V každém kroku hry vygenerujeme náhodnou hodnotou z intervalu $(0,1)$ a pokud je tato hodnota menší než hodnota $\epsilon$, tak provedeme volbu akce náhodně, v opačném případě volíme nejlepší akce dle Q-sítě.
Hodnota $\epsilon$ se na počátku inicializuje na hodnotu 1 a po každé zahrané hře se sníží vynásobením koeficientem menším než 1. 
Průběžným snižování hodnoty $\epsilon$ způsobíme, že z počátku učení se budou zkoušet náhodné akce a v průběhu přejde z prohledávání nových akcí ke zkoušení již osvědčených akcí.


\par
Při trénování se nám stává, že měníme funkci, která odhaduje Q a tím je ovlivněno i chování agenta a odhady. K zachování větší stability trénování využijeme konceptu přehrávání zkušeností (ang. Experience replay) (\cite{experienceReplay}).
Při hraní hry si v každém kroku uložíme do paměti pětici současného stavu, provedené akce, obdržené odměny, stavu, do kterého jsme se dostali, a informace zda hra neskončila.
Po konci zahrání hry následně náhodně vybereme tyto pětice z paměti a trénování provedeme na nich.



\subsection{Experiment 4: Soupeření s obranným agentem}
V prvním experimentu jsme za pomocí genetického programování hledali agenta, který je úspěšný v souboji s obranným jedincem. Pro určení jak agent v souboji obstál jsme využívali fitness funkci.
Zde, pomocí hlubokého q-učení, budeme také učit agenta vzájemnými souboji s obranným agentem, ale budeme namísto fitness funkce pro trénování využívat odměny.

Společné s prvním experimentem zde bude také stejný přístup ke vstupům a výstupům. 
Na vstupu budou opět délky všech čtyř akčních plánů a počet kroků před srážkou vesmírné lodi s asteroidem a na výstupu čtyři hodnoty reprezentující výběr konkrétního akčního plánu.

Q-učení spočívá v učení se rozhodování akcí. Akce zde v tomto pojetí však nebudou představovat elementární akce, nýbrž akční plány. 
Q-síť bude tedy volit akční plány a proto zde budeme muset provádět mezikrok pro přechod od akčních plánů k akcím.
Nejprve vždy zvolíme akční plán a následně pro pokračování v simulaci hry z vybraného akčního plánu vybereme první akci.
V tomto experimentu budeme využívat přehrávání zkušeností, tj. budeme průběžně ukládat pětice informací o přechodech do dalších stavů. 
I zde pro pamatování si zkušenosti platí, že akcí budeme rozumět akční plán.


Parametry experimentu:
\begin{itemize}    
    \item Q-síť je hustá neuronová síť s pěti vstupy, čtyřmi výstupy a jednou skrytou vrstvou. 
    \item Během učení bude zahráno 1500 her.
    \item Konstanta pro snižování $\epsilon$ je nastavena na 0.998. To znamená, že například po zahrání 1400 her se bude v další hře volit akce náhodně jen v 6\% případů. 
\end{itemize}

\par
V souboji s obranným agentem se výslednému agentovi podařilo zvítězit pouze ve čtyřech hrách. 
Nepodařilo se nám tedy sice nalézt agenta, který by stabilně porážel obranného agenta, ale dosáhli jsme jiného zajímavého výsledku.
Velkým přínosem tohoto experimentu je pestrá strategie nalezeného agenta. Výsledný agent ve velkém zastoupení používá všechny akční plány (viz \ref{Výsledek experimentu 04}). 
Výsledkem je agent, který se brání nejen sestřelováním nepřátelských asteroidů, ale i vyhýbáním se, díky tomu se agent také pohybuje a nezůstává jen staticky stát na stejném místě po celou dobu hry.
Toho se nám také podařilo dosáhnout ve 3. experimentu, ale ve srovnání s agentem získaným ze 3. experimentu je tento agent daleko více obranyschopný.




\begin{figure}[H]\centering
\includegraphics[width=145mm, height=110mm]{./Obrazky/Experiment04Results.png}
\caption{Výsledek experimentu 4}
\label{Výsledek experimentu 04}
\end{figure}



\subsection{Experiment 5: Soupeření s obranným agentem - Rozšířeno}
V předchozím experimentu jsme dosáhli zajímavého chování agenta, ale nepodařilo se nám stabilně vyhrávat nad obranným agentem.
Zkusíme proto předchozí experiment rozšířit. 
V tomto experimentu zkusíme přidat další vstupní argumenty, které by mohli agentovi pomoct v rozhodování.

Přidané parametry:
\begin{itemize}
    \item Dvojice počtu zbývajících životů obou agentů            
    \item Počet nebezpečných asteroidů v blízké vzdálenosti od agenta
    \item Celkový počet nebezpečných asteroidů v celé hře
\end{itemize}
Snaha všech přidaných argumentů je rozšířit agentovi poznání o současném stavu hry a díky tomu mu dát možnost se komplexněji rozhodovat pro akční plány.

Parametry experimentu:
\begin{itemize}
    \item Q-síť je stejná síť jako v předchozím případě, jen namísto pěti vstupních argumentů, bude nyní přijímat vstupů devět.
    \item V tomto experimentu zkusíme kvůli rozšíření vstupních argumentů také prodloužit trénování sítě, proto bude v rámci trénování zahráno 3000 her.
    \item Adekvátně ke zvýšení počtu zahraných her také zvětšíme konstantu pro snižování $\epsilon$ z hodnoty 0.998 na 0.9989. Díky tomu bude stejné pravděpodobnosti 6\% pro volbu náhodné akce dosaženo přibližně po zahrání 2550 her.
\end{itemize}



Výsledný agent dopadl velmi úspěšně. Z průběhu trénování vidíme, že agent se velmi dobře učil a od přibližně 2300. hry (viz \ref{Průběh trénování experimentu 06}) už začal vyhrávat ve větší části her.
Rozšířením vstupních argumentů a přidání trénovacích her se nám podařilo zlepšit výsledek z předchozícho experimentu.
Agent sice ztratil pestrost akčních plánů, ale za to se významně zlepšil ve vyhrávání. Z výsledku 4:10 z předchozícho experimentu se zlepšil na 10:3.
Zajimavá na nalezeném agentovi je také jeho agresivita. Agent používá útočný akční plán přibližně dvakrát tak často jako obranný plán.


\begin{figure}[H]\centering
    \includegraphics[width=145mm, height=110mm]{./Obrazky/Experiment05Results.png}
    \caption{Výsledek experimentu 5}
    \label{Průběh trénování experimentu 05}
    \end{figure}



Při testování výsledného agenta jsem si všiml, že zahrání jedné hry je časově značně náročné, přičemž to co při simulaci trvalo netriviální objem času bylo samotné dotazování Q-sítě na akční plán.
Při snaze tento problém vyřešit jsem zjistil, že v některých stavech hry jsou všechny akční plány prázdné. 
Toto může nastat v případě kdy agent stojí na místě, není ohrožený žádným asteroidem a zároveň nenalezl žádný asteroid, kterým by mohl přímo ohrozit nepřítele.
V takovém stavu nemá velký smysl rozhodovat o volbě konkrétního akčního plánu. Proto jsem nastavil, že v takových případech agent rozhodování provádět nebude.

Podobně jsem také vypozoroval, že během jedné hry často nastane situace, že právě jeden z akčních plánů je neprázdný. Překvapením pro mě bylo, že Q-síť někdy v takových případech volila jiný prázdný plán před tímto neprázdným.
Proto jsem nastavil vyjímku i pro tyto případy a v současnou chvíli platí, že když agent má k dispozici právě jeden neprázdný akční plán, tak ho volí automaticky bez dotazování se Q-sítě.
Těmito opatřeními bylo dosaženo lepší časové náročnosti hraní hry a také byl agent v souboji úspěšnější. 
Zlepšení agenta si vysvětluji právě tím, že volba jakéhokoliv neprázdného plánu před prázdným je vždy výhodnější.
Tím, že přepočítání akčních probíhá každých pár kroků, tak není pro agenta v příštím rozhodování problém tento plán opustit a začít následovat jiný.

\begin{figure}[H]\centering
    \includegraphics[width=145mm, height=110mm]{./Obrazky/Experiment05Training.png}
    \caption{Průběh trénování v experimentu 5 - Výsledné odměny jsou součtem počtu kroků hry a odměny (resp. penalizace) za výhru (resp. prohru). Samotné tyto odměny tvoří rozdíl v hodnotě 3000, díky tomu je z obrázku zřetelné, ve kterých hrách agent vyhrál.}
    \label{Průběh trénování experimentu 06}
    \end{figure}



\subsection{Experiment 6: Elementární agent proti obrannému agentovi}
V tomto experimentu zkusíme sestoupit od abstrakcí v podobě akčních plánů k elementárním akcím.
Tentokrát nebudeme Q-síť používat k volbě akčního plánu, ale přímo k volbě elementární akce.
Výsledný agent bude volit vždy jen jednu akci, proto nebudeme moci využít koncept přepočítávání akčních plánů a agent se bude muset rozhodovat v každém kroku.
Opět budeme k trénování využívat soubojů s obranným agentem a učit se na základě odměn získaných od herního prostředí.

\par
K pěti elementárním akcím (rotace vlevo, rotace vpravo, akcelerace, obyčejná střela a rozdvojovací střela), pro které se bude agent rozhodovat, přidáme navíc také možnost prázdné akce. 
Nebudeme zde volit akční plány, proto ani nemá dobrý smysl používat jejich délky jako argumenty pro rozhodování. Proto zde můžeme zvolit zcela jiný přístup.
Samotné simulace pro získání akčních plánů jsou výpočetně velmi náročné, a tedy díky tomu, že zde volíme jednodušší přístup, budeme schopni, oproti předchozím experimentům, zahrát v rámci trénování větší množství her.

\par
Jako vstupní argumenty jsem zvolil následující hodnoty:
\begin{itemize}
    \item Vektor současného pohybu vesmírné lodi
    \item Úhel natočení vesmírné lodi
    \item Počet uplynulých kroků od posledního výstřelu
    \item Relativní poloha nepřátelské lodi (vektor spojující střed vlastní lodi a střed nepřátelské lodi v její nejbližší možné poloze)
    \item Relativní polohy tří nejbližších nebezpečných asteroidů (tři vektory spojující střed vlastní vesmírné lodi se středy tří nejbližších nebezepčných asteroidů)
\end{itemize}

Parametry experimentu:
\begin{itemize}
    \item Q-síť je hustá neuronová síť s dvěmi skrytými vstvami, čtrnácti vstupními a šesti výstupními hodnotami.
    \item Díky nevyužívání akčních plánů bude hraní her rychlejší, proto pro trénování zahrajeme 10000 her.
    \item Konstanta pro snižování $\epsilon$ je nastavena na hodnotu 0.9997
\end{itemize}


Výsledný agent proti obrannému agentovi nedopadl úspěšně. V souboji byl jednoznačně poražen se skóre 0:10.
Z přehledu používaných akcí během souboje můžeme i vypozorovat proč takto dopadl. Z elementárních operací se rozhodoval v drtivé většině pro rotaci vlevo a rozdvojovací střelu.
To v praxi znamená, že se agent naučil točit dokola a kdykoliv může, tak vystřelit. Tato strategie skutečně přináší nějaké výsledky.
Touto kombinací rotace a střelby se agent dokáže ubránit před srážkou s některými asteroidy, které by ho jinak zasáhly. Zároveň tímto způsobem sestřeluje netriviální množství asteroidů, které se kolem něho nacházejí a tím potenciálně staví nepřítele do ohrožení.
Avšak pro toto chování se agent rozhoduje bezmyšlenkovitě. 
Nemíří na žádné konkrétní asteroidy, ani na nepřátelskou loď.
Největší slabinou je, že agent se zde prakticky vůbec nenaučil bránit.
Veškeré asteroidy, před kterými se agent ubrání, zasáhne díky náhodě.


\begin{figure}[H]\centering
\includegraphics[width=145mm, height=110mm]{./Obrazky/Experiment06Results.png}
\caption{Výsledek experimentu 6}
\label{Výsledek experimentu 06}
\end{figure}
    



\subsection{Experiment 7: Dva elementární agenti}
V tomto experimentu budeme podobně jako v předchozím experimentu také pracovat s agenty využívajícími pouze elementární akce.
Tentokrát ale nebudeme při trénování hrát hry proti obrannému agentovi, nýbrž proti dalšímu elementárnímu agentovi, který bude také zároveň trénován.
Budeme tedy provádět dvojí Q-učení simultánně. Cílem je zde dosáhnout vzájemného adaptivního učení, kde se každý z agentů snaží zlepšovat proti svému nepříteli a postupně tak oba agenty zlepšovat.
Agenti budou reprezentováni neuronovou sítí stejného formátu jako v předchozím experimentu.

\par


Parametry experimentu:
\begin{itemize}
    \item Každý agent bude reprezentován vlastní Q-sítí stejného formátu jako v předchozím experimentu.
    \item Tím, že pro souboj nebudeme používat obranného agenta, ale dalšího elementárního agenta, ušetříme čas na výpočtu obranného agenta. V rámci trénování tedy zahrajeme 20000 her.  
    \item Konstanta pro snižování $\epsilon$ je nastavena na hodnotu 0.9998
\end{itemize}


\begin{figure}[H]\centering
\includegraphics[width=145mm, height=110mm]{./Obrazky/Experiment07Results.png}
\caption{Výsledek experimentu 7}
\label{Výsledek experimentu 07}
\end{figure}


Výsledný souboj jsme výjimečně neprovedli proti obrannému agentovi, ale mezi vzniklými agenty mezi sebou.
Z přehledu souboje je vidět, že se agenti od předchozího experimentu nijak zásadně nezlepšili. Oba agenti se v drtivé většině stavů jen točí na jednu stranu.
Vidíme, že každý agent volí exklusivně pouze jednu stranu, na kterou se rotuje. 
Agent 1 se z přehledu souboje zdá být mírně pestřejší, kombinuje oba typy střel a navíc v téměř pětině stavů volil akceleraci.
Když jsem však vizuálně sledoval souboj agentů, tak žádný z agentů nejevil známky komplexnějšího chování. 


\subsection{Experiment 8: Souboj nejlepších nalezených agentů}
V předchozích exprimentech jsme pomocí algoritmů genetického programování a hlubokého Q-učení nalezli více různých agentů. 
V tomto experimentu již nebudeme hledat dalšího agenta, místo toho zkusíme v souboji porovnat dva agenty, reprezentující nejlepší dosažený výsledek z každého z použitých algoritmů.
A nalézt tak celkově nejlepšího agenta.
\par
Z experimentů provedených v rámci genetického programování byl v souboji s obranným agentem nejúspěšnějsí agent z druhého experimentu, ten dokázal obranného agenta porazit se skórem 10:0.
Z druhé série experimentů dopadl nejlépe agent získaný v pátém experimentu, ten také dokázal porazit obranného agenta, avšak s horším celkovým skóre 10:4.

\begin{figure}[H]\centering
\includegraphics[width=145mm, height=110mm]{./Obrazky/Experiment08Results.png}
\caption{Výsledek experimentu 8}
\label{Výsledek experimentu 08}
\end{figure}

Vybrané agenty jsme proti sobě opět nechali zápasit v souboji do deseti výher libovolného z nich.
Z výsledku souboje vidíme, že agent z druhého experimentu, který dokázal obranného agenta porazit v souboji bez jediné prohrané hry, zvítězil s velmi solidním skóre 10:2 i v tomto souboji.
Zajímavé je také, jak se agenti v tomto souboji chovali.
Agent z pátého experimentu měl v původním souboji s obranným agentem velmi agresivní přístup a volil útočný akční plán v přibližně 68\% případech.
V tomto souboji však vidíme, že podíl volby útočného plánu klesl na přibližně 54\%.
Zato agent z druhého experimentu zůstal ve svém chování naprosto konzistentní. V souboji s obranným agentem byl jeho poměr útoku k obraně 40:60 a v souboji s agentem pátého experimentu se tento poměr změnil o jediné procento na 41:59.
\par
Agenta z druhého experimentu tedy můžeme označit jako celkového nejlepšího nalezeného jedince.

\section{Zkušenost z hraní her se získanými agenty}
V předchozích experimentech jsme nalezli různé umělé agenty a ukázali jsme výsledky jejich soubojů s obranným agentem.
Nyní by bylo vhodné nahlédnout na jednotlivé agenty i z lidského subjektivního pohledu, zkusil jsem si proto se všemi zahrát pár her.
Při větší snaze se mi podařilo porazit agenty ze třetího, šestého a sedmého experimentu, tedy agenty z těch experimentů, ve kterých dopadli neúspěšně i proti obrannému agentovi.
Výjimkou byl agent z prvního experimentu, ten sice v souboji s obranným agentem prohrál, ale v zápase se mnou dokázal úspěšně vzdorovat pravděpodobně z důvodu vysokého podílu volby obranného plánu.
V zápasech se zbylými agenty, kteří dopadli v experimentech úspěšně, jsem téměř neměl šanci. 
Zde agenti byli velmi agresivní a zároveň dělali minimální počet chyb.
Z celkového pohledu mohu říct, že hraní proti všem agentům působilo velmi uměle.
Někteří agenti stojí staticky na místě a střídají útok s obranou, agenti z posledních dvou experimentů se chovají zcela chaoticky, většinu času se točí na místě a střílejí do všech stran.
Jediný agent, který se zajímavě pohyboval a choval se v jistém smyslu rozumně, byl agent ze třetího experimentu, ten ovšem nedosahuje vysoké kvality a velmi rychle ve hrách se mnou prohrál.
Měl jsem možnost si hru zahrát i s dalším člověkem a v porovnání s touto zkušenosti nebylo hraní proti umělým agentům příliš zábavné.



% 
\chapter{NEAT}

\section{Základní princip}

\section{Využití}
Kde se používá v praxi.


\section{Aplikace}
Jak jsem to použil já a jakých výsledků jsem dosáhl.




\chapter*{Závěr}
V této práci se nám úspěšně podařilo vymyslet a naimplementovat abstrakce v podobě senzorů a akčních plánů do vytvořené vesmírné hry.
Tyto abstrakce umožnili zrychlení výpočtu her a také daly agentům nástroj jak se lépe v prostředí orientovat a jak na něj vhodně reagovat.
Dále jsme se seznámili s algoritmy genetického programování a hlubokého Q-učení.
Z cílů, které jsme si v úvodu stanovili, byli tedy kompletně splněny první tři.

V závěrečné části práce jsme zmíněné algoritmy aplikovali v našem herním prostředí a provedli s nimi sérii experimentů.
V rámci nich jsme nalezli agenty, kteří se naučili bránit a útočit dostatečně dobře na to, aby pro člověka bylo téměř nemožné je porazit.
Tedy cíl nalézt agenty, kteří budou naši hru hrát velmi dobře, byl také splněn.

Bohužel se nám nepodařilo nalézt agenty, kteří by byli dobří v pohybování se po herním prostoru a to ať už za záměrem úhybného manévru, nebo z důvodu získání lepší strategické pozice.
Jako důsledek se proto agenti chovají nepřirozeně staticky a pro člověka není příliš zábavné s nimi hrát.
Proto musíme konstatovat, že cíl nalézt agenty s pestrým chováním splněn zcela nebyl.

V případě pokračování na této práci by stálo za to zamyslet se nad dalšími možnými akčními plány, obzvláště nad těmi, které by zlepšovaly pohyb vesmírné lodi po herním prostoru.
Hra může být také do budoucna lehce rozšiřitelná o další implementace agentů. Pro přidání dalšího agenta stačí implementovat jednoduchou metodu, která dostává stav hry a vrací akci a agent může být jednoduše do hry přidán. 

Kromě rozšiřování akčních plánů a přidávání dalších agentů by také mohlo být zajímavé prozkoumat další algoritmy umělé inteligence a vyzkoušet v rámci nich podobné přístupy, které byly použity v našich experimentech.

\addcontentsline{toc}{chapter}{Závěr}


%%% Seznam použité literatury
 \include{literatura}

%%% Obrázky v bakalářské práci
%%% (pokud jich je malé množství, obvykle není třeba seznam uvádět)
\listoffigures

%%% Tabulky v bakalářské práci (opět nemusí být nutné uvádět)
%%% U matematických prací může být lepší přemístit seznam tabulek na začátek práce.
% \listoftables

%%% Použité zkratky v bakalářské práci (opět nemusí být nutné uvádět)
%%% U matematických prací může být lepší přemístit seznam zkratek na začátek práce.
% \chapwithtoc{Seznam použitých zkratek}

\appendix
% \chapter{Přílohy}

% \section{První příloha}

\openright
\end{document}
